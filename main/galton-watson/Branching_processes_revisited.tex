\subsection{Revisted (skal ikke være her, men så har vi har styr på hvad der igang)}
Gej
Der er en række eksempler som kan være besværlige at håndtere med de indtil videre præsenterede værktøjer.
Eksempelvis hvis $Z_0=1$, men hvor der for hver familie gælder, at   
\begin{align*}
P(Z_1=0)<0    
\end{align*}
så er $0$ i en absorberende tilstand, og alle andre tilstand er transiente. Derfor er hele hele kæden ikkereducerbar, men  %rekurrent skal indskrives 
 når familien er $0$, så er den. Så gælder det, at $P(Z_1=0)=1$. Dog eksisterer der en stationær fordeling 
\begin{align*}
    \bs{\pi}=\begin{bmatrix}
1\\
0\\
0\\
\vdots\\
0
\end{bmatrix}
\end{align*}
da $\pi_0$ i sig selv er irreducibel.  

Dette fortæller også dog ikke noget om processen. En måde at gå til dette problem er, at kigge på opførelsen af processen ved visse hændelser med visse betingelser. 

I den resterende del af afsnittet arbejdes der med dette problem. Vi har hermed nogen faste antagelse og notationer. Der kigges på en familie af størrelse $Z_1$ med pgf og pmf 
\begin{align*}
    G(s)=E[s^{Z_1}] \quad f(k)=P(Z_1=k)
\end{align*} 
Lad generation for forgreningens død være noteret $T=\inf{n:Z_n=0}$, samt  
lad $E_n={n<T<\infty}$
være hændelserne, hvor forgreningen uddør efter. 

Vi antager også for pmf'en, at $0<f(0)+f(1)<1$ og $0<f(0)$, hvilket medfører, at 
\begin{align*}
    0<P(E_n)<1    
\end{align*}
gælder for sandsyndligen af $E_n$. Det gælder også, at $0< P(\Omega)\leq 1$

Fordelingen for $Z_n$ undersøges under betingelsen af forekomsten af $E_n$. Lad den betinget sandsynlighed af hændelsen $Z_n=j$ og $E_n$ være givet ved 
\begin{align*}
    p_j(n)=P(Z_n=j | E_n)
\end{align*}
så er grænseværdien 
\begin{align*}
q_j=\lim_{n\to \infty} p_j(n)    
\end{align*}
interessant, hvis den eksisterer, hvilket ses i følgende lemma. 

\begin{lem}
Hvis $E[X] < \infty$, så eksistere grænseværdien $\lim_{n \to \infty} p_j (n) = {}_0\pi_j$. Derudover overholdet den genererende funktion
\begin{equation*}
    G^\pi(s) = \sum_j {}_0 \pi_j s^j
\end{equation*}
ligningen 
\begin{equation} \label{alg:alg6.7.2grim}
    G^\pi\left(\frac{G_X(s\eta)}{\eta}\right) = m G^\pi(s) + 1 - m
\end{equation}
hvor $\eta = P(\Omega)$ og $m = G'(\eta)$.
\end{lem}
\begin{proof}
For $s \in [0, 1)$, lad
\begin{align*}
    G_n^\pi (s) = E(s^{Z_n} | E_n) = \sum_j {}_0p_j(n)s^j &\stackrel{(a)}= \sum_j s^j \frac{P(Z_n = j, E_n))}{P(E_n)} \\ 
    &\stackrel{(b)} = \frac{1}{\eta - G_n(0)} \sum_j s^j P(Z_n = j, E_n) \stackrel{(c)} = \frac{G_n(s\eta) - G_n(0)}{\eta - G_n(0)}
\end{align*}
hvor $G_n(s) = E(s^{Z_n})$. Lighed $(a)$ følger af ${}_0p_j(n) = \dfrac{P(Z_n = j, E_n)}{P(E_n)}$ per definition. Lighed $(b)$ af det faktum at $P(E_n) = P(\{n < T < \infty\}) = P(T < \infty) - P(T \leq n) = \eta - G_n(0)$, da $G_n(0) = P(Z_n = 0)$, og $(c)$ følger af
\begin{align*}
    P(Z_n = j, E_n) &= P(Z_n = j \text{ og alle børnenes slægter uddør }) \\
    &= P(Z_n = j) \eta^j, \text{ hvis } j \geq 1,
\end{align*}
Definer nu funktionerne 
\begin{equation*}
    H_n(s) = \frac{\eta - G_n(s)}{\eta - G_n(0)}, \quad h(s) = \frac{\eta - G(s)}{\eta - s}, \quad 0 \leq s \leq \eta
\end{equation*}
det noteres at 
\begin{equation*} \label{eq:lemmaErAids}
    1 - H_n(s \eta) = 1 - \frac{\eta - G_n(s\eta)}{\eta - G_n(0)} = \frac{\eta - G_n(0) - \eta + G_n(s\eta)}{\eta - G_n(0)} = G_n^\pi(s)
\end{equation*}
% NOTE: Hvorfor er det her vigtigt
Eftersom at $H_n$ har værdi mængde $[0, \eta)$ og $G_n^\pi$ har værdi mængde $[0, 1)$, så gælder det at 
\begin{align*}
    \frac{h(G_{n - 1}(s))}{h(G_{n - 1}(0)} \stackrel{(a)}=
    \frac{\dfrac{\eta - G_{n}(s)}{\eta - G_{n - 1}(s)}}{\dfrac{\eta - G_{n}(0)}{\eta - G_{n - 1}(0)}} &= 
    \frac{\eta - G_{n}(s)}{\eta - G_{n - 1}(s)}\frac{\eta - G_{n-1}(0)}{\eta - G_{n}(0)} \\
    &= 
    \frac{(\eta - G_{n}(s))(\eta -  G_{n - 1}(0))}{(\eta - G_n(0)(\eta - G_{n - 1}(s))}
    =
    H_n(s)H_{n - 1}(s)^{-1}
\end{align*}
hvor $(a)$ følger af proposition \ref{prop:pgfforgreningsproces}. 

Det gælder at $h$ er voksende da $G$ er konveks på $[0, \eta)$ jævnfør korollar \ref{cor:egenskaberVedPGF}, hvilket medfører er $G(s) \leq s$. Det gælder derudover per samme korollar at $G_{n - 1}$ er voksende. Hvilket giver os $H_n(s) \geq H_{n - 1}(s)$ for $s < \eta$. Derfor gælder det at grænsenværdien 
\begin{equation*}
    \lim_{n \to \infty} H_n(s\eta) = H(s\eta)
\end{equation*}
eksistere, da $H_n(s) > 0 \forall n \in N, s \in [0, \eta)$ og denne grænseværdi giver os sammen med ligning \ref{eq:lemmaErAids}, os grænseværdien at grænseværdien
\begin{equation*}
    \lim_{n \to \infty} G_n^\pi(s) = G^\pi(s)
\end{equation*}
også eksistere, for $s \in [0, 1)$. Ved at tage grænseværdien i ligning \eqref{eq:lemmaErAids} opnåes
\begin{equation}\label{alg:alg4bevislemma6.7.1grim}
   G^\pi(s) = 1 - H(s \eta), \text{ hvis } s \in [0, 1) 
\end{equation}
\textbf{Grimmit skriver at dette giver os eksistensen af ${}_0\pi_j$, men følger det ikke af at $G^\pi(s)$ eksistrere som grænse værdien til $G^\pi_n(s)$?} 

Det gælder derudover at hvis $s \in [0, \eta)$ er 
\begin{equation*}\label{eq:HnGs}
    H_n(G(s)) = \frac{\eta - G_n(G(s))}{\eta - G_n(0)} = \frac{\eta - G(G_n(0))}{\eta - G_n(0)} \cdot \frac{\eta - G_{n + 1}(s)}{\eta - G_{n + 1}(0)} = h(G_n(0))H_{n + 1}(s)
\end{equation*}
da $G(G_n(0)) = G_{n + 1}(0)$.
Bemærk at når $n \to \infty$, går $G_n(0)$ imod $\eta$ fra venstre. hvilket giver os
\begin{equation*}
    h(G_n(0)) \to \lim_{s \to \eta^+} \frac{\eta - G(s)}{\eta - s} = G'(\eta) \text{ for } n \to \infty
\end{equation*}
da $\eta = G(\eta)$, jævnfør proposition \ref{prop:prop8.7}. Lad nu $n \to \infty$ i ligning \eqref{eq:HnGs} for at opnå
\begin{equation} \label{alg:alg6bevislemma6.7.1grim}
    H(G(s)) = G'(\eta)H(s), \text { hvis } s \in [0, \eta)
\end{equation}
Benyt ligning \ref{alg:alg4bevislemma6.7.1grim} til at opnå
\begin{equation*}
    G^\pi(\eta^{-1}G(s)) + 1
\end{equation*}
Tilsidst \textbf{todo med ligningen}
\end{proof}

\begin{rem}
Når $\mu=E[Z_1] \leq 1$, så er det sikkert, at forgreningen uddør ifølge proposition \ref{prop:prop8.9}, således er $P(\Omega)=1$, samt $G'(P(\Omega))=\mu$. Dette leder til, at \eqref{alg:alg6.7.2grim} kan reduceres til 
\begin{align*}
    G^\pi(G(s))=\mu (G^\pi(s)-\mu)+1
\end{align*}
\end{rem}
\begin{cor} \label{cor:cor6.7.7grim}
Lad $\mu=E[Z_1]$. Hvis 
\begin{itemize}
    \item $\mu \neq 1$, så er $\sum_{j=0}q_j=1$. 
    \item $\mu=1$, så er $q_j=0$ for alle $j$. 
\end{itemize}
\end{cor}
\begin{proof}
Bevis af $\mu=1$:\\
Vi har, at $G'(P(\Omega))=1$ hvis og kun hvis $\mu=1$ fra bemærkningen. Hermed kan \eqref{alg:alg6.7.2grim} omskrives til \begin{align*}
    G^\pi(G(s))=G^\pi(s)
\end{align*}
da $P(\Omega)=1$, når $\mu\leq1$. Fra proposition \ref{prop:prop8.7} har vi, at hvis $\mu > 1$, så er $G(s)>s$ for alle $s<1$. Og da $G^\pi(G(s)) =G^\pi(s)$, og følgen $\{G_n(0)\}_{n \geq 0}$ konvergere imod $1$, da $G(1) = 1$ og talfølgen er voksende, har vi at $G^\pi(s)=G^\pi(0)=0$ for alle $s<1$. 

Hermed er $G^\pi(s)=\sum_j q_j s^j =0$, hvilket medfører, at
\begin{align*}
    q_j=0 \quad \text{for alle j}
\end{align*}

Bevis af $\mu\neq1$:\\ 
Modsat første tilfælde, så gælder det, at $\mu\neq1$ hvis og kun hvis $G'(P(\Omega))\neq 1$. Lad $s\to P(\Omega)$ i \eqref{alg:alg6bevislemma6.7.1grim}, så fåes
Lad ${s_n}_{n \geq 0}$ være en følge som konvergere imod nede fra $\eta$, da $H_n(s_n)$ konvergere uniformt har vi

\begin{align*}
    \lim_{s\to \eta^-}H(s)= \lim_{n \to \infty} H_n(s_n) = \lim_{n \to \infty} \frac{\eta - G_n(s_n)}{\eta - G_n(0)} = 0
\end{align*}
eftersom $G_n(s_n) \to \eta$, når $n \to \infty$.

Derfor fåes det fra \eqref{alg:alg4bevislemma6.7.1grim}, at
\begin{align*}
    \lim_{s\to 1} G^\pi(s)=\lim_{s\to 1} 1-H(sP(\Omega))=1 
\end{align*}
da $\lim_{s\to 1} H(sP(\Omega))=0$. Hermed
er
\begin{align*}
    \lim_{s\to 1} G^\pi(s)=\lim_{s\to 1} \sum_j q_js_j =1
\end{align*}
hvor $\sum_j q_j=1$
\end{proof}

Det er svært at undersøge forgrenningsprocceserne, hvor $\mu=1$ og $j\leq 1$, da
\begin{align}
    P(Z_n=j)\to 0 \quad \text{ forgreningen uddør}
    \\
    P(Z_n=j |E_n) \to 0 \quad \text{da } Z_n\to\infty \text{ under betingelse af } E_n 
\end{align}
Lad nu $\sigma^2=\Var[Z_1]$ og  
\begin{align*}
Y_n=\frac{Z_n}{n\sigma^2}    
\end{align*}
Så er det muligt at vise fordelingen for $Y_n$ betinget af $E_n$ er konvergent, når $n\to\infty$.  

\begin{thm} \label{thm:thm6.7.8grim}
Hvis $\mu=E[Z_1]$ og $G''(1)<\infty$, så overholder $Y_n$, at
\begin{align*}
    \lim_{n\to\infty}P(Y_n\leq y|E_n) = 1-e^{-2y}
\end{align*}
\end{thm} 
\begin{proof}

\end{proof}









