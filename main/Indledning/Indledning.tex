I 1800-tallets England var der en bekymring for, at de adliges efternavne uddøde. Dette undersøgte Galton Watson sammen med Henry William Watson, hvor de kom frem til de såkaldte Galton-Watson processer. Måden man praktisk kigger på om en families navn uddør, er ved hjælp af deres stamtræ, hvor man kan udregne sandsyndlighedenfor om familiens navn på et tidspunkt uddør. Dette kunne på det historiske tidspunkt kun ske hvis ingen sønner blev født.

\quad Selvom problemet startede med kun at handle omkring de adliges efternavne, så kan man også anvende teorierne på forgreninger. I vores projekt vil der bliver sat fokus på coronavirussen, der var i mellem "2019-2022". Her vil vi ved hjælp af teorier omkring forgreningsprocesser og diskrete fordelinger undersøge simulationerner for forgreningerne for at finde sandsynligheden for, at epidamien ville uddø. Før vi vil kommme frem til dette, så vil vi introducere  nødvendig teori for at forstår problemet. Dette består af generel viden om sandsynlighedsregning, forgreningsprocesser, som bygger på Markov kæder og så simulering af diskrete variabler. 
