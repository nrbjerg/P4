I 1800-tallets England var der stor prestige forbundet med ens efternavn. Denne prestige var der dog risiko for at dø ud, med familienavnet. Sandsynlighedn for at et efternavn uddør undersøgte Francis Galton sammen med Henry William Watson, hvor de kom frem til de såkaldte Galton-Watson processer. I disse processer beregner man ved hjælp af slægters stamtræer,  sandsyndligheden for om familiens navn på et tidspunkt uddør. Dette kunne på det historiske tidspunkt kun ske hvis ingen sønner blev født.

\quad Selvom teorien startede med kun at handle omkring de adliges efternavne, så kan man også anvende teorierne på forgreninger, det kan være alt fra celledelinger til fission, på atomkerne niveau. I vores projekt vil der bliver sat fokus på coronavirussen, hvor Galton-Watson processer bliver brugt til at simulere fremtidige smittetal. Her vil vi ved hjælp af teorier omkring forgreningsprocesser og diskrete fordelinger undersøge simulationerner for forgreningerne for at finde sandsynligheden for, at epidamien ville uddø. Der vil blive introduceret  nødvendig teori for at forstår problemerne, og hvilker væktøjer der bruges til at behandle dem. Dette består af generel viden om sandsynlighedsregning, simulerering af diskrete stokastiske variabler og forgreningsprocesser, som bygger på teorien om Markov kæder. 
