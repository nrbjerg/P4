\section{Flere stokastiske variable}
I tidligere afsnit har vi fokuseret på enkelte diskrete stokastiske variabler. Vi vil i dette afsnit introducere teori, til tilfælde med  flere stokastiske variable.
\begin{defn} \label{def:def3.8bogen}%def3.8
    Lad $X_1, X_2, \ldots$ være diskrete stokastiske variable. Så kaldes de \textbf{uafhængelige}, hvis hændelserne $X_k \in A_k$ for $k \in \N$, er uafhængelige hændelser, for alle $A_1, A_2, \ldots \subseteq \R$, således $A_k \neq \emptyset \; \forall k \in \N$.
\end{defn}

Da de bagvedliggende hændelser er uafhængelige har vi også at 
\begin{align*}
    P\left(\bigcup^n_{k = 1} X_k \in A_k\right)= \prod^n_{k = 1}P(X_k\in A_k)
\end{align*}
for alle $A_1, A_2, \ldots, A_n \subseteq \R$, såfremt $X_1, X_2, \ldots X_n$ er uafhængelige.

\begin{prop} \label{prop:prop3.9} %prop 3.9
    Lad $X$ og $Y$ være diskrete stokastiske variable med pmf henholdsvis $p_X$ og $p_Y$. Så er $X$ og $Y$ uafhængige hvis og kun hvis 
    \begin{align*}
        P(X = x, Y = y)=p_X(x)p_Y(y) 
    \end{align*}
for alle $x,y\in \R$
\end{prop}
\begin{proof}
    Antag at $X$ og $Y$ er uafhængige. Lad $A=\{x \}$ og $B=\{y \}$ så er
    \begin{align*}
        P(X \in A,Y \in B)=P(X \in A)P(Y \in B)= P(X = x)P(Y = y) = p_X(x)p_Y(y)
    \end{align*}
Modsat, antag nu at $P(X = x, Y = y) = p_X(x)p_Y(y)$ og betragt $A\subseteq \R$ og $B\subseteq \R$, så gælder  
\begin{align*}
    P(X\in A, Y\in B)=\sum_{x\in A}\sum_{y\in B}P(X = x, Y =y) =\sum_{x\in A}p_X(x)\sum_{y\in B}p_Y(y)=P(X\in A)P(Y\in B)
\end{align*}
og hermed er $X$ og $Y$ uafhængige.
\end{proof}
I vise tilfælde har vi flere variabler, som følger samme fordeling, derfor introduceres følgende definition
\begin{defn} %3.19
    Lad $X_1,X_2,\ldots$ være uafhængige stokastiske variabler med samme fordeling. Så kaldes variablerne  \textbf{uafhængige og identisk fordelte (iid)}.
\end{defn}
\begin{exmp}
    Lad $X_1, X_2, \ldots, X_6$ være diskrete stokastiske variable, og lad $X_k$ beskrive antallet af 6'er ved det $k$'te kast af en fair terning, så er $X_1, X_2, \ldots, X_6$ uafhængelige, jævnfør \ref{exmp:uafhængelighed}, samtidig er 
    \begin{equation*}
        p_{X_k}(x) = \begin{cases}
            \frac{5}{6}, \; x = 0\\
            \frac{1}{6}, \; x = 1\\
            0, \text{ ellers}
        \end{cases}
    \end{equation*} 
    for $k = 1, 2, \ldots, 6$. $X_1, X_2, \ldots X_6$ er altså uafhængige og identisk fordelte   
\end{exmp}

Når der arbejdes med mere end én variabel viser følgende definition sig at være nyttig.
\begin{defn} % definition 3.3 & 3.4
    Lad $n \in \N$ og $X_1, X_2, \ldots, X_n$ være diskrete stokastiske variable og lad $X_j$ have udfaldsrum $\{x_{j, 1}, x_{j, 2}, \ldots\}$. Så kaldes $(X_1, X_2, \ldots, X_n)$ en diskret stokastisk vektor og funktionen
    \begin{align*}
        p(x_{1, k_1}, x_{2, k_2}, \ldots, x_{n, k_n}) = P(X_1 = x_{1,k_1}, X_2 = x_{2, k_2}, \ldots, X_n = x_{n, k_n})
    \end{align*}
    for \textbf{simultan pmf'en} af $(X_1, X_2, \ldots, X_n)$, og pmf'en for $X_j$ kaldes en \textbf{marginal pmf}.
    %Hvis $X$ og $Y$ er diskrete stokastiske variabler, så kaldes $(X,Y)$ for en diskrete stokastiske vektor. Hvis vektoren har udfaldsrummet ${(x_j,y_k) : j,k=1,2\ldots}$, så er funktionen 
    %\begin{align*}
    %    p(x_j,y_k)=P(X=x_j,Y=y_k)
    %\end{align*}
    %den \textbf{simultan pmf} af $(X,Y)$. Den en-dimensionel pmf af henholdsvis $X$ og $Y$ \textbf{marginal pmf'en}
\end{defn}
\begin{cor}
Lad $(X, Y)$ være en diskret stokastisk vektor, så er $X$ og $Y$ uafhængelige hvis og kun hvis
\begin{equation*}
    p(x, y) = p_X(x)p_Y(y)
\end{equation*}
for alle $x, y \in \R$.
\end{cor}
\begin{proof}
Da $p(x, y) = P(X = x, Y = y)$, følger resultatet direkte af proposition \ref{prop:prop3.9}.
\end{proof}


\begin{exmp} \label{examp:tern_vektor}
    Lav to terningkast. Lad $X$ være udkommet af det første terningkast og $Y$ være udkommet af det andet. 
    Udfaldsrummet for de to kast er $\{(x, y) | x, y \in \N, 0 < x, y \leq 6\}$. Så er den simultane pmf for et par $\{x_k,y_k\}$ givet ved
    \begin{align*}
        p(x_j,y_k)=P(X=x_j,Y=y_k)=P(X=x_j)P(Y=y_k)=\frac{1}{6}\frac{1}{6}=\frac{1}{36}
    \end{align*}
Fra eksempel \ref{exp:terning} vides det, at de to diskrete stokastisk variabler er uafhængige.
\end{exmp}

De marginale pmf findes ud fra den simultane pmf på følgende måde. 
\begin{prop}\label{prop:marginalPmf} %prop:3.2
    Hvis den diskrete stokastiske vektor $(X,Y)$ har den simultan pmf $p$, så er marginal pmf'erne af $X$ og $Y$ givet ved
    \begin{align*}
        p_X(x_j)=\sum_{k=1}^\infty p(x_j,y_k), \quad & j=1,2\ldots
        \\
        p_Y(y_k)=\sum_{j=1}^\infty p(x_j,y_k), \quad & k=1,2\ldots
    \end{align*}
\end{prop}
\begin{proof}
Da hændelsen $\{X=x_j\}$ er ækvivalent med $\{X=x_j,Y \in S\}$, så er 
\begin{align*}
    p_X(x_j)=\sum_{k=1}^\infty P(X=x_j,Y=y_k)=\sum_{k=1}^\infty p(x_j,y_k)
\end{align*}
Beviset er analogt for $Y$. 
\end{proof}

\begin{exmp}
    Fortsættelse af  eksempel \ref{examp:tern_vektor}. Det ses, at marginal pmf'en for $X$ er 
    \begin{align*}
        p_X(x_j)=\sum_{k=1}^6 p(x_j,y_k)=\sum_{k=1}^6 p_X(x_j)p_Y(y_k) = p_X(x_j) \sum_{k=1}^6 p_Y(y_k)
        =\frac{1}{6}
    \end{align*}
    Da $\sum^6_{k = 1} p_Y(y_k) = 1$.
\end{exmp}

Hvis vi undersøgere den forventede værdi af en linearkombination af to diskrete stokastiske variable, får vi en linearkombination af forventede værdier.
\begin{prop}\label{prop:forventedVærdiAfLinearKombination} %Cor:3.3
Lad $X, Y$ være diskrete stokastiske variable med udfaldsrummene $\{x_1, x_2, \ldots \}$ og $\{y_1, y_2, \ldots \}$ og $c_X, c_Y \in \R$, så er
\begin{equation*}
    E[c_XX+c_yY] = c_XE[X] + c_YE[Y]
\end{equation*}


\end{prop}
\begin{proof}
Vi har
\begin{align*}
    E[c_X X + c_Y Y] = \sum^\infty_{j = 1} \sum^\infty_{k = 1} (c_X x_j + c_Y y_k) p(x_j, y_k) &= \sum^\infty_{j = 1} \sum^\infty_{k = 1} c_X x_j p(x_j, y_k) + \sum^\infty_{j = 1} \sum^\infty_{k = 1} c_Y y_k p(x_j, y_k) \\
    &\stackrel{(a)}= \sum^\infty_{j = 1} \sum^\infty_{k = 1} c_X x_j p(x_j, y_k) + \sum^\infty_{k = 1} \sum^\infty_{j = 1} c_Y y_k p(x_j, y_k) \\
    &\stackrel{(b)}= \sum^\infty_{j = 1} c_X x_j p_X(x_j) + \sum^\infty_{k = 1} c_Y y_k p_Y(y_k)\\
    &= E[c_X X] +  E[c_Y Y] \stackrel{(c)}= c_X E[X] + c_Y E[Y]
\end{align*}
Hvor vi har benyttet, at 
\begin{equation*}
    \sum^\infty_{k = 1} \sum^\infty_{j = 1} c_Y y_k p(x_j, y_k) = \sum^\infty_{k = 1} c_Y y_k p(y_k) = c_YE[Y]
\end{equation*}
medfører, at vi kan bytte rundt på summerings rækkefølge for at opnå ligheden $(a)$, jævnfør \cite{horiaPowerSerieNoter}.
Lighederne $(b)$ og $(c)$ følger af proposition \ref{prop:marginalPmf} og \ref{prop:bogen prop2.21}.
\end{proof}

Proposition \ref{prop:forventedVærdiAfLinearKombination} gælder også for tilfældet med $n$ forskellige diskret stokastisk variabler.
\begin{cor}\label{cor:sumAfDiskreteVariable} %Prop 3.14
Lad $X_1, X_2, \ldots, X_n$ være diskrete stokastiske variabler og $c_1, c_2, \ldots, c_n \in \R$, så er 
\begin{align*}
    E\left[\sum^n_{k = 1} c_k X_k\right] = \sum^n_{k = 1} c_k E[X_k]
\end{align*}
\end{cor}

\begin{proof}
    Lad $Y_m = \sum_{k = m}^n c_k X_k$ for $m = 1, 2, \ldots, n - 1$, så er $Y_m$ en diskret stokastisk variabel, og vi kan benytte \ref{prop:forventedVærdiAfLinearKombination} til at lave følgende omskrivning, hvis $m \neq n - 1$
    \begin{align*}
        E\left[Y_m\right] = E\left[c_m X_m + Y_{m + 1} \right] &= c_m E[X_m] + E[Y_{m + 1}] 
    \end{align*}
    og for $m = n - 1$ har vi
    \begin{equation*}
        E[Y_{n - 1}] = E[c_n X_n] = c_N E[X_n]
    \end{equation*}
    jævnfør \ref{prop:bogen prop2.21}. Heraf fåes 
    \begin{equation*}
        E\left[\sum^n_{k = 1} c_k X_k\right] = E[Y_1] = c_1 E[X_1] + E[Y_2] = \cdots = \left(\sum^{n - 1}_{k = 1} c_k E[X_1]\right) + E[Y_{n - 1}] = \sum^n_{k = 1} c_k E[X_1]
    \end{equation*}
    hvilket afslutter vores bevis.
\end{proof}
\begin{exmp} %ex 3.21
    Hvor mange gange skal man slå med en terning for at have slået alle muligheder?
    Første slag må givetvis give et brugbart resultat. For at nå anden brugbare resultat, er der  $\frac{5}{6}$ sandsynlighed for at få et brugbart resultat, det tredje brugbare resultat fremkommer ved $\frac{4}{6}$, og så fremledes. Dette giver følgende formel
    \begin{align*}
        X=1+\sum^{6}_{k=2}X_k
    \end{align*}
    Hvor $X_k\sim \geom\left(\frac{7-k}{6}\right)$
    Det giver at 
    \begin{align*}
        E[X]=1+\sum^{6}_{k=2}E[X_k] = 1+ \left(\frac{6}{5}+\frac{6}{4}+\dots+\frac{6}{1}\right) \approx 14,7
    \end{align*}
    jævnfør proposition \ref{prop:geomEgenskaber}. Altså forventes det at tage lidt under $15$ slag for at have opnået alle forskellige udfald af en terning.
\end{exmp}

Den forventede værdi af en diskret stokastisk vektor indsat i en funktion $g(X,Y)$ kan findes på nærmest samme måde som i proposition \ref{prop 2.12} og beviset er også analogt. 
\begin{prop} \label{prop:prop3.11}%prop 3.11
    Lad $(X,Y)$ være en diskret stokastisk vektor med simultan pmf $p$, samt lad $g:\R \times \R \to \R$ være en funktion, så er 
    \begin{align*}
        E[g(X,Y)]=\sum_{j=1}^\infty \sum_{k=1}^\infty g(x_j,y_k)p(x_j,y_k)
    \end{align*}
\end{prop}
\begin{proof}
Lad $V = g(X,Y)$ og $v_{j,k} = g(x_j,y_k)$, så er pmf $p_V(v_{j,k}) = p(x_j,y_k)$, og eftersom 
$E[V] = \sum_{j=1}^\infty \sum_{k=1}^\infty v_{j,k} p_V(v_{j,k})$
følger det at $E[g(X,Y)] = \sum_{j=1}^\infty \sum_{k=1}^\infty g(x_j,y_k) p(x_j,y_k)$.
\end{proof}

Den forventede værdi af diskrete stokastiske variabler kan også opdeles, når det kommer til multiplikation.  
\begin{lem} \label{lem:prop3.13} %prop 3.13
    Lad $X$ og $Y$ være uafhængige diskrete stokastiske variable, så er 
    \begin{align*}
        E[XY]=E[X]E[Y]
    \end{align*}
\end{lem}
\begin{proof}
    Da $X$ og $Y$ er uafhængige, så har vi fra proposition \ref{prop:prop3.9}, at $p(x,y)=p_X(x)p_y(y)$. Ved hjælp af proposition \ref{prop:prop3.11} fåes
    \begin{align*}
        E[XY]=\sum_{j=1}^\infty\sum_{k=1}^\infty xyp(x_j,y_k)&=\sum_{j=1}^\infty\sum_{k=1}^\infty x_jy_kp_X(x_j)p_Y(y_k)\\ &=\sum_{j=1}^\infty x_jp_X(x_j)\sum_{k=1}^\infty y_kp_Y(y_k)
    \end{align*}
    men dette er blot definitionen på den forventede værdi så $E[XY] = E[X]E[Y]$
\end{proof}

\begin{prop} \label{prop:prop3.15} %prop 3.15
    Hvis $X_1,\ldots, X_n$ er uafhængige diskrete stokastiske variable, så er
\begin{align*}
    E[X_1X_2\cdots X_N]=E[X_1]E[X_2]\cdots E[X_N]
\end{align*}
\end{prop}
\begin{proof}
Ved hjælp af proposition \ref{lem:prop3.13} fås
    \begin{align*}
        E[X_1X_2\cdots X_n]
        &=E[X_1X_2\cdots X_{n-1}]E[X_n]
        \\
        &=E[X_1,X_2,\cdots X_{n-2}]E[X_{n-1}]E[X_n]\\
        &\; \vdots \\
        &=E[X_1]E[X_2]\cdots E[X_{n-1}]E[X_n]
    \end{align*}
\end{proof}

\begin{prop}\label{prop:variansAfX+Y}
Lad $X, Y$ være uafhængelige diskrete variable, så er $\Var[X + Y] = \Var[X] + \Var[Y]$.
\end{prop}
\begin{proof}
  Ved benyttelse propoisition \ref{prop:forventedVærdiAfLinearKombination} og lemma \ref{lem:prop3.13} har vi at
  \begin{align*}
    \Var[X + Y] &= E[(X + Y)^{2}] - E[(X + Y)]^{2} \\
                &= E[X^{2} + 2XY + Y^{2}] - (E[X] + E[Y])^{2} \\
                &= E[X^{2}] + E[2XY] + E[Y^{2}] - E[X]^{2} + E[Y]^{2} + 2E[X]E[Y]\\
                &= E[X^{2}] + E[Y^{2}] - E[X]^{2} + E[Y]^{2}
  \end{align*}
  men dette er blot definitionen på varians, så $\Var[X + Y] = \Var[X] + \Var[Y]$.
\end{proof}


\begin{cor} \label{cor:variansAfLinearKombinationAfUafhængigeVariable}
  Lad $X_{1}, X_{2}, \ldots, X_{n}$ være uafhængige diskrete variable, og $c_{1}, c_{2}, \ldots, c_{n}$, så er
  \begin{equation*}
    \Var\left[\sum^{n}_{k = 1} c_{k}X_{k}\right] = \sum_{k=1}^n c_{k}^{2} \Var[X_{k}]
  \end{equation*}
\end{cor}
\begin{proof}
  Definier $Y_{k} = c_{k}X_{k}$. Så følger det ved gentagende benyttelse af
  \ref{prop:variansAfX+Y} at
  \begin{equation*}
    \Var\left[\sum_{k=1}^n Y_{k}\right] = \sum_{k=1}^n \Var[Y_{k}] = \sum_{k=1}^n \Var[c_{k}X_{k}] = \sum_{k=1}^n c_{k}^{2}\Var[X_{k}].
  \end{equation*}
  hvor sidste lighed følger af proposition \ref{prop:prop2.15}.
\end{proof}
