\section{Stokastisk Variable} \label{sec:SV}
\begin{defn}%def 2.1 og 2.2
En reel variabel med værdi fra et stokastisk eksperiment kaldes en \textbf{stokastisk variabel} $X$. Værdimængden af $X$, kaldes \textbf{udfaldsrummet} af $X$. Hvis udfaldsrummet af $X$ er tællelig kaldes $X$ en \textbf{diskret stokastisk variabel}.
\end{defn}

Vi husker, at en hændelse afhængig af stokastiske variabler kan noteres med et prædikat, så som
\begin{align*}
    \{X=0\}=\Set*{s\in S\given X=0}
\end{align*}
der beskriver hændelsen, hvor $X$ antager værdien $0$.
Generelt undlades tuborg klammerne omkring predikatet, når der arbejdes med sandsynlighedsmål, så $P(\{X = 0\})$ noteres også $P(X = 0)$.

\begin{defn} [Sandsynlighedsfunktion]%Def 2.3
    Lad $X$ være en diskret stokastisk variabel med udfaldsrum $V_X$. Så kaldes funktionen
\begin{align*}
    p(x)=
    \begin{cases}
        P(X=x) & x\in V_X\\
        0 & x\not\in V_X
    \end{cases}
\end{align*}
for \textbf{sandsynlighedsfunktion (pmf)} af $X$. Sandsynlighederne, beskrevet ved $p(x_k)$, kaldes \textbf{fordelingen} af $X$.
\end{defn}
Vi introducerer nu følgende resultat, som kan benyttes til at generalisere resultaterne om sandsynlighedsfunktioner til diskrete stokastiske variable med uendelige udfaldsrum til diskrete stokastiske variabler med endelige udfaldsrum.
\begin{lem}
Lad $X$, være en diskret stokastisk variabel med udfaldsrum $\{x_1, x_2, \ldots, x_n\}$, og $f: \N \to \R$. Så er 
\begin{equation*}
    \sum^n_{k = 1} f(k) p(x_k) = \sum^\infty_{k = 1} f(k) p(x_k)
\end{equation*}
hvor $x_k \in \R \backslash \{x_1, x_2, \ldots, x_n\}$ for $k = n + 1, n + 2, \ldots$
\end{lem}
\begin{proof}
Vi har 
\begin{align*}
    \sum^\infty_{k = 1} f(k) p(x_k) =\sum^n_{k = 1} f(k) p(x_k) + \sum^\infty_{k = n + 1} f(k) p(x_k) = \sum^n_{k = 1} f(k) p(x_k) 
\end{align*}
da $p(x_k) = 0$ for alle $x_k \not \in \{x_1, x_2, \ldots, x_n\}$.
\end{proof}



% En sandsynlighedsfunktion kaldes skrives på engelsk probability mass function, heraf forkortelsen \textbf{pmf} fremkommer, som vi herefter vil benytte.

\begin{prop}\label{prop:kravTilPMF}\label{prop:pmfEgenskaber} %Prop:2.1
Lad $X$ være en stokastisk variabel med udfaldsrum $\{x_1, x_2\ldots\}$ og pmf $p$, så er 
\begin{enumerate}
    \item $p(x_k) \geq 0$ for $k=1,2 \ldots$
    \item $\displaystyle \sum_{k=1}^{\infty} p(x_k) = 1 $
\end{enumerate}
\end{prop}
\begin{proof}
Da $P$ er et sandsynlighedsmål, medfører det, at $0 \leq p(x_k) = P(X = x_k) \leq 1 \text{ for } k=1,2,\ldots$. Derudover gælder, at $P(S)=1$, da $S = \bigcup^\infty_{k = 1} \{X=x_k\}$ og $\{X=x_i\} \cap \{X=x_j\} = \emptyset$ for $i \neq j$, gælder det, at
\begin{equation*}
    1 = P(S) = \sum^\infty_{k = 1} P(X=x_k) = \sum_{k=1}^\infty p(x_k)
\end{equation*}
\end{proof}
I stedet for kun at kigge på hændelser, hvor $\{ X=x \}$, kan man yderligere se på hændelser, hvor $\{X \leq x \}$. Dette kan eksempelhvis være sandsynligheden for at slå mindre end eller lig med et bestemt tal $x$ på et terningkast. 
\begin{defn} %2.4
Lad $X$ være en stokastiske variable. Funktionen $F: \R \to \R$ givet ved
\begin{align*}
    F(x)=P(X\leq x), 
\end{align*}
kaldes den \textbf{kumulative fordelingsfunktion (cdf)} af $X$.
\end{defn}

Ved hjælp af proposition \ref{prop:sandsynlighedMedToHændelser} punkt \ref{enu:propsandsynlighedMedToHændelser1}, kan man omskrive tilfældet $P(X\geq x)$ til $P(X\geq x)=1-P(X<x)$. Yderligere gælder følgende egenskaber for cdf'er. 

\begin{prop}%2.2 COPY PASTE
Lad $X$ være en diskret stokastisk variable med udfaldsrum $\{x_1, x_2, \ldots\}$, pmf, $p$ og cdf $F$. Så gælder
\begin{enumerate}
    \item $F(x)=\displaystyle \sum_{k:x_k \leq x} p(x_k)$
    \item $p(x_k)=F(x_k)- \displaystyle \lim_{y \rightarrow x_k^- } F(y)$ for $k= 1,2, \ldots $
    \item hvis $B \subseteq \R$ er $P(X \in B)= \displaystyle \sum_{k:x_k \in B} p(x_k)$.
\end{enumerate}
\end{prop} 
\begin{proof}
Først bevises punkt 3. Lad $X_V$ være udfaldsrummet af $X$. Vi har derved, at
\begin{align*}
    P(X\in B)=\sum_{r\in B}p(r)=\sum_{r\in(B\cap X_V)}p(r)=\sum_{k:x_k\in B}p(x_k)
\end{align*}
hvilket beviser punkt 3. Dette anvendes til at bevise punkt 1,
\begin{align*}
    F(x)=P(X\in(\infty,x])=\sum_{k:x_k\in ]-\infty,x]}p(x_k)=\sum_{k:x_k\leq x}p(x_k)
\end{align*}
Til sidst bevises punkt 2 ved
\begin{align*}
    F(x_k)-\lim_{y\rightarrow x_k^-}F(y)=P(X\leq x_k)-P(X<x_k)=P(X=x_k)=p(x_k)
\end{align*}
\end{proof}
