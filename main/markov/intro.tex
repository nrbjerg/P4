Dette afsnit bygger på \cite{sandsynlighedsBog} og \cite{grimsandsynlighedsBog}
\begin{defn}
Lad $\{X_n\}_{n = 0}^\infty$, være en følge af diskrete stokastiske variable, udfaldsrum $S$. Så kaldes $\{X\}_{n = 0}^\infty$ en diskret \textbf{Markov kæde} hvis, den overholder \textbf{Markov betingelsen}:
\begin{equation*}
    P(X_n = s \;| X_0 = x_0, \ldots, X_{n - 1} = x_{n - 1}) = P(X_n = s \;| X_{n - 1} = x_{n - 1})
\end{equation*}
for alle $n \in \N$ og $s, x_0, \ldots x_{n - 1} \in S$. Og i såfald kaldes $X_k$ \textbf{udfaldet} af Markov kæden i tiden $k$.
\end{defn}
\begin{exmp} Lad følgen af diskrete stokastiske variable $\{S_{n}\}_{n \geq 0}$, være definieret $S_{n} = \sum_{i=0}^n X_{i}$ for $n \in \N_{0}$, hvor $X_{i}$ er iid. diskrete stokastiske variable $P(X_{i} = 1) = p$ og $P(X_{i} = -1) = 1 - p$. Så er $\{S_{n}\}_{n \geq 0}$ et eksempel på en markov kæde \footnote{Faktisk er $\{S_{n}\}_{n \geq 0}$ et eksempel på hvad der kaldes en tilfældig tur}, da
  \begin{equation*}
    S_{n} = S_{n - 1} + X_{i} \implies P(S_{n} = k_{n} | S_{0} = k_{0}, \ldots, S_{n - 1} = k_{n - 1}) = P(S_{n} = k_{n} | S_{n - 1} = k_{n - 1})
  \end{equation*}
  for alle $k_{0}, k_{1}, \ldots, k_{n} \in \Z$.
\end{exmp}
\begin{defn}
Markov kæden $\{X_n\}_{n = 0}^{\infty}$ kaldes \textbf{homogen} hvis
\begin{equation*}
  \infty
    P(X_n = i | X_{n - 1} = j) = P(X_1 = i | X_0 = j)
\end{equation*}
for alle $n \in \N$ og $i, j \in S$.
\end{defn}
Vi vil fremadrettet antage at alle Markov kæder er homogene, medmindre andet eksplicit defineres.
\begin{defn}
Lad $\{X_n\}_{n = 0}^\infty$ være en Markov kæde så defineres \textbf{transpositions matricen} $P_X = [p_{ij}]$, hvor 
\begin{equation*}
    p_{ij} = P(X_1 = j \;| X_0 = i).
\end{equation*}
og \textbf{$n$ skridts tranpositions matricen} $P_X(n) = [p_{ij}(n)]$ hvor
\begin{equation*}
    p_{ij}(n) = P(X_n = j \;| X_0 = i)
\end{equation*}
\end{defn}

\begin{example}

\end{example}


\begin{thm}
Lad $\{X\}_{n = 0}^\infty$ være en Markov kæde, så er $P_X(n) = P_X^n$ for $n \in \N$.
\end{thm}
\begin{proof}
Vi beviser dette ved induktion, for $n = 1$ gælder ligheden per definition. Antag nu at $P_X(n - 1) = P_X^{n - 1} = [p_{ij}(n - 1)]$ så er 
\begin{align*} 
    p_{ij}(n) = P(X_n = j \;| X_0 = i) &\stackrel{(a)}= \sum_{k \in S} P(X_n = j\; | X_{n - 1} = k, X_0 = i) P(X_{n - 1} = k\; | X_0 = i) \\
    &\stackrel{(b)}= \sum_{k \in S} P(X_n = j\; | X_{n - 1} = k) P(X_{n - 1} = k\; | X_0 = i) \\
    &= \sum_{k \in S} p_{jk}p_{ki}(n - 1) 
\end{align*}
Hvor lighed $(a)$ følger af \ref{thm:LTP}, $(b)$ af Markov betingelsen. Men $\displaystyle \sum_{k \in S} p_{jk}p_{ki}(n - 1)$ er blot indgang $i, j$ i matrix produktet $P_X P_X(n - 1) = P_X P_X^{n - 1} = P_X^n$, per vores induktions antagelse.
\end{proof}

\begin{defn}
Et udfald $i \in S$ kaldes \textbf{rekurrent} hvis 
\begin{equation*}
    P(X_n = i \text{ for et } n \in \N \;| X_0 = i) = 1
\end{equation*}
hvis $P(X_n = i \text{ for et } n \in \N \;| X_0 = i) < 1$ kaldes udfaldet \textbf{transient}.
\end{defn}

\begin{example}
  Lad
\end{example}

Som det kunne ses i eksemplet var det ikke muligt at forlade $0$, vi introducere derfor følgende definition
\begin{definition}
Hvis $P(X_{n} = i \text{ for et } n \in \N \;| X_{0} = j) = 0$ for alle $i \neq j$, så kaldes udfaldet $j$ for \textbf{absorberende}.
\end{definition}
\begin{example}
  Det ses nemt at ethvert absorberende udfald også er et rekurrent udfald.
\end{example}



\begin{defn}
Lad $\{X_n\}_{n = 0}^\infty$ være en Markov kæde og $i, j \in S$. $i$ og $j$ siges at \textbf{kommunikere}, skrevet $i \to j$, hvis $\exists m \in \N_0$ således $p_{ij}(m) > 0$. Hvis $i \to j$ og $j \to i$ siges $i$ og $j$ at \textbf{intrakommunikere}, skrives $i \leftrightarrow j$.\\ Lad $C \subseteq S$.
\begin{enumerate}[i)]
    \item Så kaldes $C$ \textbf{lukket} hvis $p_{ij} = 0$ for alle $i \in C, j \not \in C$. 
    \item og hvis  $i \leftrightarrow j$ for alle $i, j \in C$ kaldes $C$ \textbf{irreducibel}.
\end{enumerate}
\end{defn}

\begin{defn}
Lad $\{X_n\}_{n = 0}^\infty$ være en Markov kæde. En vektor $\bs{\pi}$, med indgange $\pi_j \in \R^+$ for $j \in S$, kaldes en \textbf{stationær} fordeling af $\{X_n\}_{n = 0}^\infty$, hvis $\bs{\pi}P_X = \bs{\pi}$ og $\displaystyle \sum_{j \in S} \pi_j = 1$.
\end{defn}
Fordelingen kaldes stationær fordi $\bs{\pi} P_X^n = \bs{\pi} P_X^{n - 1} = \cdots = \bs{\pi} P_X = \bs{\pi}$.
