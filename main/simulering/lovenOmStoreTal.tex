\section{Loven om store tal}
\begin{theorem} \label{thm:Markovsulighed}[Markovs ulighed]
Hvis $X$ er en stokastisk variable, således
$E[X] < \infty$. Så gælder
\begin{align*}
    P(|X|\geq a) \leq \frac{E[|X|]}{a} \text{ for alle } a > 0
\end{align*}
\end{theorem}
\begin{proof}
Lad $A=\{|X|\geq a\}$, så fremstilles indikator funktionen af $A$, som $I_A$. Så tages den forventede værdi på begge sider, hvorefter man kan flytte $a$ uden for.
\begin{align*}
E[|X|]\geq E[aI_A] = a E[I_A] \iff \frac{E[|X|]}{a}\geq E[I_A]
\end{align*}
hvor vi har benyttet at $E[I_A] = P(A)$.
\end{proof}
Det er værd at bemærke, at hvis $a<0$ så skal relationen vendes om.

Chebyshews ulighed siger noget om, hvad sandsynligheden er for, at en stokastisk variabel ligger inden for $c$ gange standardafvigelser for middelværdien.
\begin{theorem} \label{Thm:Chebyshewsulighed}[Chebyshews ulighed]
    Lad $X$ være en stokastisk variable med middelværdien $\mu$ og variansen $\sigma^2>0$. For alle $c>0$ gælder, at
    \begin{align*}
        P(|X-\mu|\geq c \sigma)\leq \frac{1}{c^2}
    \end{align*}
\end{theorem}
\begin{proof}%fra wiki
    \begin{align*}
        P(|X-\mu|\geq c \sigma)=P((X-\mu)^2\geq c^2\sigma^2)
    \end{align*}
Ved at benytte Markovs ulighed, sætning \ref{thm:Markovsulighed} opnåes
\begin{align*}
    P((X-\mu)^2\geq c^2\sigma^2) \leq \frac{E[(X-\mu)^2]}{c^2\sigma^2}
\end{align*}
Da $E[(X-\mu)^2]$ er defineret af $\sigma^2$, så indsættes dette
\begin{align*}
    \frac{E[(X-\mu)^2]}{c^2\sigma^2} = \frac{\sigma^2}{c^2\sigma^2} = \frac{1}{c^2}
\end{align*}
Hvilket afslutter beviset.
\end{proof}

Loven om store tal beskriver, at ved en test med et stort antal forsøg vil den emperiske middelværdi, nærme sig den teoretiske middelværdi, jo flere forsøg der bliver foretaget.
\begin{thm} \label{thm:law_of_large_numbers}%theorem 4.1
    Lad $X_1, X_2, \dots$ være en følge af iid. med stokastisk variabler, med middelværdien $\mu$, of lad $\Bar{X}$ være den emperiske middelværdi. For hver $\varepsilon>0$ gælder at
    \begin{align*}
        P(|\Bar{X}-\mu|>\varepsilon) \rightarrow 0 \text{ når } n \rightarrow \infty
    \end{align*}
\end{thm}
\begin{proof}
    Antag at $X_k$ har en afgrænset varians, dvs $\sigma^2<\infty$. Anvend Chebyshews ulighed sætning \ref{Thm:Chebyshewsulighed}ulighed på $\Bar{X}$, og lad $c=\varepsilon \sqrt{x}/\sigma$. Eftersom $E[\Bar{X}]=\mu$ og $\Var[\Bar{X}]=\sigma^2/n$, dermed giver det
    \begin{align*}
        P(|\Bar{X}-\mu| > \varepsilon) \leq \frac{\sigma}{n\varepsilon^2} \rightarrow 0 \text{ når } n \rightarrow \infty
    \end{align*}
\end{proof}
