% cmds.tex : brugerdefinerede makroer og blokke
% ------------------------------------------------------------------------------
% Denne fil indeholder definitioner for egne makroer og blokke, der bruges som
% genvejsfunktioner for ofte brugte kommandoer og tekst i rapporten.

% Matematiske symboler ---------------------------------------------------------

\newcommand{\N}{\mathbb{N}}
\newcommand{\Z}{\mathbb{Z}}
\newcommand{\Q}{\mathbb{Q}}
\newcommand{\F}{\mathbb{F}}
\newcommand{\R}{\mathbb{R}}
\newcommand{\C}{\mathbb{C}}
\newcommand{\ind}{\mathbbm{1}}
\newcommand{\bigO}{\mathcal{O}}
\newcommand{\mb}[1]{\mathbf{#1}}
\newcommand{\e}{\mathrm{e}}
\newcommand{\range}{\text{range}}
\newcommand{\Span}{\text{span}}
\newcommand{\Null}{\text{null}}

% Sætninger o.lign. ------------------------------------------------------------
% http://www.ctex.org/documents/packages/math/amsthdoc.pdf
\theoremstyle{plain}                 % Titel med fed, tekst med skråskrift
\newtheorem{thm}{Sætning}[chapter]   % Sætninger, nummereret efter kapitel
\newtheorem{lem}[thm]{Lemma}         % Lemmaer, nummereret ligesom sætninger
\newtheorem{prop}[thm]{Proposition}  % Propositioner, nummereret ligesom sætninger
\newtheorem{cor}[thm]{Korollar}         % Korollarer, unnumbered
\newtheorem{theorem}[thm]{Sætning}   % Sætninger, nummereret efter kapitel
\newtheorem{lemma}[thm]{Lemma}         % Lemmaer, nummereret ligesom sætninger
\newtheorem{proposition}[thm]{Proposition}  % Propositioner, nummereret ligesom sætninger
\newtheorem{corollary}[thm]{Korollar}         % Korollarer, unnumbered

\theoremstyle{definition}            % Title med fed, opretstående tekst
\newtheorem{DEFN}[thm]{Definition}   % Definitioner, nummereret ligesom sætninger
\newtheorem{EXMP}[thm]{Eksempel}      % Eksempler, nummereret ligesom sætninger

\theoremstyle{remark}
\newtheorem*{rem}{Bemærk}
\newtheorem*{remark}{Bemærk}
% Figur-makroer ----------------------------------------------------------------

% imgfig ("image figure")
% Makro til at indsætte et billede fra fig/img-mappen
% Argumenter:
%   * (valgfri) figurbredde; procent af sidebredde (standard: 0.75)
%   * filnavn (uden fig/img/ eller filendelse); også brugt til label
%   * figurteksten
% Eksempler:
%   \imgfig{filnavn}{Figurteksten skrives her}
%   \imgfig[0.5]{filnavn}{Figurteksten skrives her}
\newcommand{\imgfig}[3][0.75]{
  \begin{figure}[htbp]
    \centering
    \includegraphics[width=#1\textwidth]{fig/img/#2}
    \caption{#3}
    \label{fig:#2}
  \end{figure}
}

% dimgfig ("double image figure")
% Makro til at indsætte to billeder ved siden af hinanden
% Argumenter:
%   * (valgfri) breddefordeling (standard: 0.5, dvs. lige fordeling)
%   * filnavn for den venstre figur, uden fig/img/ eller filendelse
%   * billedtekst for den venstre figur
%   * filnavn for den højre figur, uden fig/img/ eller filendelse
%   * billedtekst for den højre figur
% Eksempler:
%   \dimgfig{billede1}{Første billedtekst}{billede2}{Anden billedtekst}
%   \dimgfig[0.3]{billede1}{Første billedtekst}{billede2}{Anden billedtekst}
% Alterativt, se
% https://en.wikibooks.org/wiki/LaTeX/Floats,_Figures_and_Captions#Subfloats
\newcommand{\dimgfig}[5][0.5]{
  \ifx\dimgleftwidth\undefined
    \newlength{\dimgleftwidth}
    \newlength{\dimgrightwidth}
  \fi
  \setlength{\dimgleftwidth}{#1\textwidth-0.02\textwidth}
  \setlength{\dimgrightwidth}{0.96\textwidth-\dimgleftwidth}
  \begin{figure}[htbp]
    \centering
    \begin{minipage}[t]{\dimgleftwidth}
      \centering
      \includegraphics[width=\linewidth]{fig/img/#2}
      \caption{#3}
      \label{fig:#2}
    \end{minipage}
    \hfill
    \begin{minipage}[t]{\dimgrightwidth}
      \centering
      \includegraphics[width=\linewidth]{fig/img/#4}
      \caption{#5}
      \label{fig:#4}
    \end{minipage}
  \end{figure}
}

% Mængdebygger notation
% ANVENDELSE: \Set*{a} eller \Set*{a \given b} (Stjernen sørger for resizing af brackets osv.)
\newcommand{\given}{\nonscript\,\delimsize\vert\nonscript\,\mathopen{}}
\DeclarePairedDelimiterX{\Set}[1]{\{}{\}}{#1}

%Replace emptyset
\renewcommand{\emptyset}{\text{Ø}}
\newcommand{\Var}{\textnormal{Var}}
\newcommand{\unif}{\textnormal{unif}}
\newcommand{\geom}{\textnormal{geom}}
\newcommand{\Poi}{\textnormal{Poi}}
\newcommand{\bs}[1]{\boldsymbol{#1}}
\newcommand{\norm}[1]{\lvert\lvert#1\rvert\rvert}
\newcommand{\abs}[1]{\lvert#1\rvert}
\renewcommand{\qedsymbol}{$\blacksquare$}
